\documentclass{beamer}
\usepackage[utf8]{inputenc}
\usepackage[spanish]{babel}
\usepackage{beamerthemeshadow}
\usepackage{graphicx}
\usepackage{latexsym}
\usepackage[font=Times,timeinterval=30, timeduration=30.00, timedeath=0,timewarningfirst=80, timewarningsecond=90]{tdclock}
\usepackage{multirow}
\usepackage[T1]{fontenc} % fix some warnings with fonts
\usepackage{fix-cm}      % fix some warnings with fonts

 \useoutertheme{infolines}

\usefonttheme[onlymath]{serif}
\usefonttheme[onlysmall]{structurebold}
 
\usepackage{color} 

\setbeamertemplate{navigation symbols}{}
\usetheme{Darmstadt}
\usecolortheme{orchid}
\beamersetuncovermixins{\opaqueness<1>{25}}{\opaqueness<2->{15}}

%\usetheme{Berlin}
%\useoutertheme{infolines}
%\setbeamercovered{dynamic}

%-----------------------------------------
% Definición de comandos útiles
%-----------------------------------------
\newcommand{\RR}{{\mathbb{R}}}
\newcommand{\NN}{{\mathbb{N}}}
\newcommand{\ZZ}{{\mathbb{Z}}}
\newcommand{\vs}{{\vspace{5mm}}}
\newcommand{\vts}{{\vspace{15mm}}}
\newcommand{\hs}{{\hspace{5mm}}}
\newcommand{\hts}{{\hspace{15mm}}}
\newcommand{\vv}[1]{{\mathbf{#1}}}
\newcommand{\expected}[2][\!]{\:\mathop{\mathcal{E}}\limits_{#1}\!\left[{#2}\right]}
\renewcommand{\vec}[1]{\boldsymbol{\mathbf{#1}}}

\title[]{Most Discriminative Atom Selection for Apnea-Hypopnea Events Detection}
%
\author[]{R.E. Rolon \inst{1} \and L.E. Di Persia \inst{1,2} \and H.L. Rufiner \inst{1,2,3} \and R.D. Spies \inst{2,4}}
\institute[]{\inst{1} Instituto de I+D en Se\~{n}ales, Sistemas e Inteligencia Computacional, sinc($i$), UNL-CONICET. Santa Fe, Argentina. \and \inst{2} Consejo Nacional de Investigaciones Cient\'{i}ficas y T\'{e}cnicas, CONICET. Argentina. \and \inst{3} Laboratorio de Cibern\'{e}tica, Fac. de Ingenier\'{i}a, Univ. Nacional de Entre R\'{i}os. Entre R\'{i}os, Argentina. \and \inst{4} Instituto de Matem\'{a}tica Aplicada del Litoral, IMAL. Santa Fe, Argentina.}


\makeatletter
\setbeamertemplate{footline}
{
	\leavevmode%
	\hbox{%
		\begin{beamercolorbox}[wd=.3\paperwidth,ht=2.25ex,dp=1ex,center]{date in head/foot}%
			\usebeamerfont{date in head/foot}sinc($i$), UNL-CONICET - IMAL~~\beamer@ifempty{\insertshortauthor}{}{}
		\end{beamercolorbox}%
		\begin{beamercolorbox}[wd=.4\paperwidth,ht=2.25ex,dp=1ex,center]{author in head/foot}%
			\usebeamerfont{author in head/foot}CLAIB 2014
		\end{beamercolorbox}}%
		\begin{beamercolorbox}[wd=.3\paperwidth,ht=2.25ex,dp=1ex,right]{date in head/foot}%
			\usebeamerfont{date in head/foot}\insertshortdate{}\hspace*{2em}
			\insertframenumber{} / \inserttotalframenumber\hspace*{2ex}
		\end{beamercolorbox}%
		\vskip0pt%
	}
	\makeatother

\date[30/10/14]{\tiny October 30, 2014}

\logo{\includegraphics[height=3mm, angle=90]{unl}}

\AtBeginSubsection[]
{
  \begin{frame}<beamer>
    \tableofcontents[currentsection,currentsubsection]
  \end{frame}
}

\begin{document}
%----------------------------------
\begin{frame}
%\initclock
\titlepage
\end{frame}

\begin{frame}
 \tableofcontents
\end{frame}

%---------------------------------------
\section{Introduction}
%---------------------------------------
\subsection{Motivation}
%---------------------------------------
\begin{frame}
\frametitle {Motivation}
\begin{block}{}
	\begin{itemize}
		\item The Obstructive Sleep Apnea-Hypopnea (OSAH) syndrome is characterized by repetitive episodes of airway narrowing during sleep.
		\pause
		\item The OSAH has a prevalence of 2\% to 4\% in the adult population.
		\pause
		\item The current gold standard diagnostic test for OSAH is an overnight polysomnography (PSG) in a sleep laboratory.
	\end{itemize}
\end{block}
\end{frame}
%--------------------------------------- 
  \begin{frame}
\frametitle {Medical criteria}
\begin{itemize}
\item \alert{\textit{apnea:}} if the amplitude of the airflow signal decreases below 25\% of the ``baseline'' breathing amplitude and it remains below that level for more than 10 seconds.\\
\pause
\item \alert{\textit{hypopnea:}} if the amplitude of the respiratory signal decreases below 70\% of the ``baseline'' breathing amplitude, it remains so for more than 10 seconds for more than 2 breathe periods.\\
\end{itemize}
\pause
Apnea-Hypopnea Index (AHI) = average number of AH events per hour.
\begin{itemize}
	\item 5$<$AHI$\leq$15, mild.
	\item \alert{15$<$AHI$\leq$30}, moderate.
	\item \alert{AHI$>$30}, severe.
\end{itemize}
\end{frame}
%--------------------------------------- 
\begin{frame}
\frametitle {Polysomnography}
  \begin{center}
        %\includegraphics[width=0.75\columnwidth,trim=0 0cm 0 0cm,clip]{imagenes/psgb1} 
  \end{center}
  \end{frame}
%---------------------------------------
\section{Materials and methods}
%---------------------------------------
\subsection{Proposed method}
%---------------------------------------
\begin{frame}
\frametitle {Proposed method}
\begin{center}
	%\includegraphics[height=6.5cm]{imagenes/Diagrama1.eps}
\end{center}
\end{frame}
%---------------------------------------
\subsection{Database}
%---------------------------------------
\begin{frame}
\frametitle {SHHS: Sleep Heart Health Study}
The SHHS database contains 1000 PSGs of the ``Sleep and Epidemiology Research Center (SERC)\footnote{http://cci.case.edu/serc/}'' at the ``Case Western Reserve University''.\\
	\pause
\begin{itemize}
	\item Biomedical signals:
	\begin{itemize}
	 	\item Nasal airflow
	 	\item \alert{$\text{SaO}_2$}
   	 	\item OXStat
   	 	\item EEG
	\end{itemize}
	\pause
   \item Expert annotations:
	\begin{itemize} 
	 \item \alert{Respiratory events (AH)}
 	 \item Sleep stages
	\end{itemize}
  \end{itemize}
\end{frame}
%---------------------------------------
\begin{frame}
\frametitle {Signals of interest}
\begin{center}
% \includegraphics[height=7cm]{imagenes/sen_int.eps}
\end{center}
\end{frame}
%---------------------------------------
\subsection{Sparse representations}
%---------------------------------------
\begin{frame}
\frametitle {Dictionary}
\vspace{-5mm}
\begin{center}
\begin{table}
	\vspace{2mm}
\begin{tabular}{c l}
	%\vspace{2mm}
%\multirow{4}{*}{\centering\includegraphics[width=65mm]{./imagenes/diccionario1}} 
\vspace{2mm}
& $\vec{s}=\sum_{j=1}^{M}\vec{\phi}_j a_j=\vec{\Phi} \vec{a}$\\
\vspace{2mm}
& $\vec{s} \in \RR^N$\\
\vspace{2mm}
& $\vec{\Phi} \in \RR^{N \times M}$, $M\geq N$\\
\vspace{2mm}
& $\vec{a} \in \RR^M$\\
\end{tabular}
\end{table}
\end{center}
\vspace{1.8cm}
\pause
Sparse representation problem:
\begin{itemize}
	\item \textit{learning}.
	\item \textit{inference}.
\end{itemize}
\end{frame}
%---------------------------------------
\subsection{Learning and inference problems}
%---------------------------------------
\begin{frame}
	\frametitle {Learning and inference}
	\begin{itemize}
		\item Noise Overcomplete Independent Component Analysis (NOCICA)
		\begin{equation}
			\vec{s}=\sum_{j=1}^{M}\vec{\phi}_j a_j+\vec{\varepsilon}=\Phi \vec{a}+\vec{\varepsilon}.
		\end{equation}
		Then:
		\begin{equation}
			\Delta \vec{\Phi}= \eta \varLambda_{\varepsilon} ((\vec{s}-\vec{\Phi} \vec{a}_{MAP})\vec{a}_{MAP}^T-\vec{\Phi} H^{-1}).
		\end{equation}
		\pause
		\item Orthogonal Matching Pursuit (OMP)
		\begin{equation}
		\text{min} {||\vec{s}-\vec{\Phi}\vec{a}||}_{2} \text{ subject to } {||\vec{a}||}_{0}\leq T,
		\end{equation}
		where ${||\cdot||}_{0}$ denotes the zero-norm.
	\end{itemize}
\end{frame}
%---------------------------------------
\section{MDAS method}
%---------------------------------------
\begin{frame}
\frametitle {Most Discriminative Atom Selection}
\begin{block}{}
	\begin{center}
		The idea behind this method is to select the most discriminative atoms of $\vec{\Phi}$ in order to improve the classifier's performance.\\
	\end{center}
\end{block}
Main steps:\\
\begin{enumerate}
\item Compute the atom activation frequency $n_{ci}^j$ given the class $i$ and the atom $j$.\\
\item Select the most discriminative atoms of $\vec{\Phi}$ by $D=|n_{c1}^j-n_{c2}^j|$.\\
\end{enumerate}
\vspace{-3mm}
\begin{figure}
%\includegraphics[height=3.7cm]{imagenes/abs_dif_freq_act_oad.png}
\end{figure}
\end{frame}
%---------------------------------------
\begin{frame}
\frametitle {Most Discriminative Atom Selection}
\begin{block}{}
	\begin{center}
		The idea behind this method is to select the most discriminative atoms of $\vec{\Phi}$ in order to improve the classifier's performance.\\
	\end{center}
\end{block}
Substeps:
\begin{enumerate}
\item Improve the neural network performance.\\
\item Obtain the optimal configuration of the classifier.\\
\end{enumerate}
\begin{figure}
%	\includegraphics[height=4cm]{imagenes/MDAS_plot.png}
\end{figure}
\end{frame}
%---------------------------------------
\section{Experiments and results}
%---------------------------------------
\begin{frame}
\frametitle {Database}
Training set:
\begin{center}
\begin{tabular}{c c}
\hline
AHI & Total studies \\
\hline
AHI$\leq$5 & 5\\
%\hline
5$<$AHI$\leq$10 & 5\\
%\hline
10$<$AHI$\leq$15 & 5\\
%\hline
AHI$>$15 & 5\\
\hline
\end{tabular}
\end{center}
\pause
Test set:
\begin{center}
\begin{tabular}{c c}
\hline
AHI & Total studies \\
\hline
AHI$\leq$5 & 21\\
%\hline
5$<$AHI$\leq$10 & 21\\
%\hline
10$<$AHI$\leq$15 & 21\\
%\hline
AHI$>$15 & 21\\
\hline
\end{tabular}
\end{center}
\end{frame}
%---------------------------------------
\begin{frame}
\frametitle {AH events detection}
\begin{figure}
 %  \includegraphics[height=6cm]{imagenes/AHevd1.eps}
\end{figure}
\end{frame}
%---------------------------------------
\begin{frame}
\frametitle {Scatter plots}
\begin{figure}
  % \includegraphics[height=4.5cm]{imagenes/scatter_plot_mdas_oad.eps}
   %\hspace{-0mm}
  % \includegraphics[height=4.5cm]{imagenes/scatter_plot_mdas_cd.eps}
  \end{figure}
\end{frame}
%---------------------------------------
\begin{frame}
\frametitle {Tables of results}
MDAS method (Multilayer perceptron):
\begin{center}
\begin{tabular}{c c c}
\hline
  & OAD & CD \\
\hline
Inputs & 24 & 30\\
%\hline
Neurons (hidden layer) & 14 & 14\\
\hline
\end{tabular}
\end{center}
\pause
Studies:
\begin{center}
\begin{tabular}{c c c}
\hline
  & OAD & CD \\
\hline
Total studies & 84 & 84\\
%\hline
Sensibility (\%) & 74.52 & 68.86\\
%\hline
Specificity (\%) & 76.73 & 67.69\\
%\hline
Correlation (\%) & 90.04 & 74.57\\
\hline
\end{tabular}
\end{center}
\end{frame}
%---------------------------------------
\section{Discussion and conclusions}
%---------------------------------------
\begin{frame}
\frametitle {Discussion and conclusions}
     \vspace{-5mm}
  \begin{itemize}
   %\item OAD and CD dictionaries were used to obtain sparse representations of an $SaO_2$ signal.
   %\pause
   \item A novel methodology for detecting AH events by using only the $\text{SaO}_2$ signal was developed.
   \pause
   \item MDAS method selects the most discriminative atoms of $\vec{\Phi}$.   
   \pause
   \item A reasonable NN performance by using a subset of atoms of $\vec{\Phi}$ was obtained.
   \pause
   \item A considerably high AHIest-AHI correlation by using OAD was obtained.
   \pause
   \end{itemize}
\end{frame}
%---------------------------------------
\begin{frame}
 \frametitle{}
 \begin{center}
    \vspace{-5mm}
  \begin{beamercolorbox}[center,shadow=true,rounded=true]{title}
    \usebeamerfont{title}¡Thank you!\\
  \end{beamercolorbox}
  \vspace{20mm}
  \footnotesize Contact: \emph{Román Rolon}\\
%  \email{romanrolon@gmail.com}
\scriptsize \texttt{rrolon@santafe-conicet.gov.ar}\\
  \vspace{2cm}
 % \includegraphics[scale=1.2]{imagenes/sinc}\hspace{1mm}
 % \includegraphics[height=5.8mm]{imagenes/fichunl}\hspace{1mm}
 % \includegraphics[height=6.2mm]{imagenes/Logo_IMAL}\hspace{1mm}\\
 \end{center}
\end{frame}
%---------------------------------------
\end{document}

%%-------------------------------
%%\section*{Recordar}
%%\begin{frame}
%% Especificidad: probabilidad de clasificar correctamente un individuo sano$$\frac{VN}{VN + FP}$$
%% Sensibilidad: probabilidad de clasificar correctamente un individuo enfermo$$\frac{VP}{VP + FN}$$
%% Curvas ROC: es una curva de la sensibilidad en función de la especifidad, que permite comparar la performance de distintos detectores.
%%\end{frame}


